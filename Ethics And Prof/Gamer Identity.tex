% Please do not change the document class
\documentclass[11pt]{scrartcl}

% Please do not change these packages
\usepackage[hidelinks]{hyperref}
\usepackage[none]{hyphenat}
\usepackage{setspace}
\doublespace

% You may add additional packages here
\usepackage{amsmath}

% Please include a clear, concise, and descriptive title
\title{Should Game Developers be Responible for the Fragmentation of Gamer Identities and Culture?}

\date{November 14, 2016}

% Please do not change the subtitle
\subtitle{COMP230 - Ethics and Professionalism}

% Please put your student number in the author field
\author{1605913}



\begin{document}

\maketitle

\section{Introduction}

Video game culture does not have a strict definition not much research has been filled out in terms of cultural studies. The definition of cultural studies is a number of approaches that are under constant conflict over what the definitions, methods, and theories of cultural studies are\cite{shaw2010video}. Although the games industry borrows techniques from various disciplines such as philosophy and economics it has not drawn as deep as deep as it might from cultural studies especially in not being critical and reflexive. The author believes that is why there is such a lack of diversity in the games industry\cite{shaw2010video}.

\section{What Is a Gamer?}

To first look at whether game developers should be responsible for the fragmentation of gamer identities, we must first look at the definition of a "Gamer" this sort of definition is central to studying video game culture \cite{shaw2010video}. Many studies disprove the misconception that the "Typical Gamer" is a white, heterosexual, male teen \cite{JCC4:JCC4428}. Gender is also a huge area of study on gamer identities where studies are trying to find multiple ways to include girl gamers where the media accept that "boys" and "girls" have different play styles. In turn, this should supposedly make video game culture more accessible to female gamers.\cite{cassell2000barbie}


\section{Broadening Diversity}

Articles have acknowledged that since the \emph{Nintendo Wii's} release gaming has been brought to a larger audience \cite{schiesel_2007}. Which may prove true William's prediction that the media covering a more diverse gaming audience would create a more diverse gaming audience. \cite{williams2003video} The expansion of broader types of gaming is not done for the benefit of the community it is used mainly for marketers and developers to make more money.\cite{elliott_2005} Which is counterintuitive as "minority" players such as women, males of many ages, "non-gamers" and those of a different racial and cultural background which represent a majority of the population.\cite{fron2007hegemony} Due to what the common demographic for games is, a large quantity of research centered on players is male-centered.\cite{fron2007hegemony}


\section{Game Developers}

One of the sources of gamers fragmented culture is the game developers themselves, for example, a games journalist had a game played for her because the industry PR rep did not believe she had the ability to play the game that he was advertising saying, "I think I better play it for you"\cite{consalvo2012confronting}. A point that is consistently raised by developers and those in defense of developers is that the games industry is driven by the market for them.\cite{fron2007hegemony} The author thinks that a reason for the industry having a hard time creating games for women is that they do not employ a large number of women in industry according to figures from Skillset in 2009 women only represented 4\% of the games industry workforce.\cite{skillset_2009} This looks at developmental roles of women in the industry, which deal with the creative and developmental aspect of game development. This also influences who the target market will be of the games that they work on.\cite{prescott2011segregation} Which may go to show why not many games are aimed towards women or girls. Within the industry, it has been shown to be a "boys-only" ethos such as when gaming producer \emph{Nour Polloni} wanted a female character to wear baggy pants but the all-male creative team wanted to dress her in a string bikini. The author thinks that this is one of the reasons that women are not being heard within the industry because their ideas are silenced, thus they do not want to nor can advance in industry\cite{marriott_2003}. On the other hand game developers such as \emph{Bioware}have stood against sexist attitudes to some players thus giving women more of a voice and making them want to participate in the industry.\cite{consalvo2012confronting}


\subsection{Statistics}

Table 1 of \cite{prescott2011segregation} shows the percentage of men and women in each job role within the games industry. Even though the Entertainment Software Association(ESA) reported that 38\% of gamers are women, this goes against the conception that the games industry follows the market\cite{fron2007hegemony}. A study created by AOL on casual gamers found that women over 40 spent a large quantity of time playing casual games.\cite{pearce2008truth} Even with these statistics, many in the games industry do not consider "Casual Gamers" to be real gamers. While other companies have chosen to ignore "minority" gamers, \emph{Nintendo} has chosen to do the opposite and create adverts that show a variety of players such as women, girls, adults and baby boomers playing with each other, rather than showing off fancy graphics\cite{surowiecki_2017}.The author feels that if other game developers decided to work to this principle of advertising to larger audiences they would see an increase in sales, where a well-designed personalized advert will be more persuasive to the customer than a traditional billboard\cite{shannon2009profiling}. Through the use of IOS and Android devices, women and girls will be regularly found playing games such as \emph{Angry Birds}. Microsoft have even intended to increase their audience fan base through the use of the Kinect and games such as \emph{Dance Central} and indie games such as journey do not have a "preferred" gendered player.\cite{consalvo2012confronting} 

\section{Conclusion}
In conclusion, the games industry has a very big responsibility in changing how women and "minority" gamers are given a voice within and outside the industry. As most game developers that have been looked at in this essay are still run by a "boys-only" ethos\cite{marriott_2003}. Developers should be looking to follow \emph{Nintendos} example of advertising to a wide range of audiences, not just the "hardcore" gamer profile. Once game developers have started to adopt these principles, the rest should follow such as PR reps and those working game developers realizing that women can play games.



\bibliographystyle{ieeetran}
\bibliography{references}

\end{document}